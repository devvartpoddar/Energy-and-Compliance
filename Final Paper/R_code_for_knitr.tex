\documentclass[twoside]{article}
\usepackage{lmodern}
\usepackage{amssymb,amsmath}
\usepackage{ifxetex,ifluatex}
\usepackage{fixltx2e} % provides \textsubscript
\ifnum 0\ifxetex 1\fi\ifluatex 1\fi=0 % if pdftex
  \usepackage[T1]{fontenc}
  \usepackage[utf8]{inputenc}
\else % if luatex or xelatex
  \ifxetex
    \usepackage{mathspec}
  \else
    \usepackage{fontspec}
  \fi
  \defaultfontfeatures{Ligatures=TeX,Scale=MatchLowercase}
  \newcommand{\euro}{€}
\fi
% use upquote if available, for straight quotes in verbatim environments
\IfFileExists{upquote.sty}{\usepackage{upquote}}{}
% use microtype if available
\IfFileExists{microtype.sty}{%
\usepackage{microtype}
\UseMicrotypeSet[protrusion]{basicmath} % disable protrusion for tt fonts
}{}
\usepackage[margin=1in]{geometry}
\usepackage{hyperref}
\PassOptionsToPackage{usenames,dvipsnames}{color} % color is loaded by hyperref
\hypersetup{unicode=true,
            pdftitle={Policy Reverberation and Electricity Prices in the United States},
            pdfauthor={Emma Heffernan \& Devvart Poddar},
            pdfborder={0 0 0},
            breaklinks=true}
\urlstyle{same}  % don't use monospace font for urls
\usepackage{natbib}
\bibliographystyle{plainnat}
\usepackage{color}
\usepackage{fancyvrb}
\newcommand{\VerbBar}{|}
\newcommand{\VERB}{\Verb[commandchars=\\\{\}]}
\DefineVerbatimEnvironment{Highlighting}{Verbatim}{commandchars=\\\{\}}
% Add ',fontsize=\small' for more characters per line
\usepackage{framed}
\definecolor{shadecolor}{RGB}{248,248,248}
\newenvironment{Shaded}{\begin{snugshade}}{\end{snugshade}}
\newcommand{\KeywordTok}[1]{\textcolor[rgb]{0.13,0.29,0.53}{\textbf{{#1}}}}
\newcommand{\DataTypeTok}[1]{\textcolor[rgb]{0.13,0.29,0.53}{{#1}}}
\newcommand{\DecValTok}[1]{\textcolor[rgb]{0.00,0.00,0.81}{{#1}}}
\newcommand{\BaseNTok}[1]{\textcolor[rgb]{0.00,0.00,0.81}{{#1}}}
\newcommand{\FloatTok}[1]{\textcolor[rgb]{0.00,0.00,0.81}{{#1}}}
\newcommand{\ConstantTok}[1]{\textcolor[rgb]{0.00,0.00,0.00}{{#1}}}
\newcommand{\CharTok}[1]{\textcolor[rgb]{0.31,0.60,0.02}{{#1}}}
\newcommand{\SpecialCharTok}[1]{\textcolor[rgb]{0.00,0.00,0.00}{{#1}}}
\newcommand{\StringTok}[1]{\textcolor[rgb]{0.31,0.60,0.02}{{#1}}}
\newcommand{\VerbatimStringTok}[1]{\textcolor[rgb]{0.31,0.60,0.02}{{#1}}}
\newcommand{\SpecialStringTok}[1]{\textcolor[rgb]{0.31,0.60,0.02}{{#1}}}
\newcommand{\ImportTok}[1]{{#1}}
\newcommand{\CommentTok}[1]{\textcolor[rgb]{0.56,0.35,0.01}{\textit{{#1}}}}
\newcommand{\DocumentationTok}[1]{\textcolor[rgb]{0.56,0.35,0.01}{\textbf{\textit{{#1}}}}}
\newcommand{\AnnotationTok}[1]{\textcolor[rgb]{0.56,0.35,0.01}{\textbf{\textit{{#1}}}}}
\newcommand{\CommentVarTok}[1]{\textcolor[rgb]{0.56,0.35,0.01}{\textbf{\textit{{#1}}}}}
\newcommand{\OtherTok}[1]{\textcolor[rgb]{0.56,0.35,0.01}{{#1}}}
\newcommand{\FunctionTok}[1]{\textcolor[rgb]{0.00,0.00,0.00}{{#1}}}
\newcommand{\VariableTok}[1]{\textcolor[rgb]{0.00,0.00,0.00}{{#1}}}
\newcommand{\ControlFlowTok}[1]{\textcolor[rgb]{0.13,0.29,0.53}{\textbf{{#1}}}}
\newcommand{\OperatorTok}[1]{\textcolor[rgb]{0.81,0.36,0.00}{\textbf{{#1}}}}
\newcommand{\BuiltInTok}[1]{{#1}}
\newcommand{\ExtensionTok}[1]{{#1}}
\newcommand{\PreprocessorTok}[1]{\textcolor[rgb]{0.56,0.35,0.01}{\textit{{#1}}}}
\newcommand{\AttributeTok}[1]{\textcolor[rgb]{0.77,0.63,0.00}{{#1}}}
\newcommand{\RegionMarkerTok}[1]{{#1}}
\newcommand{\InformationTok}[1]{\textcolor[rgb]{0.56,0.35,0.01}{\textbf{\textit{{#1}}}}}
\newcommand{\WarningTok}[1]{\textcolor[rgb]{0.56,0.35,0.01}{\textbf{\textit{{#1}}}}}
\newcommand{\AlertTok}[1]{\textcolor[rgb]{0.94,0.16,0.16}{{#1}}}
\newcommand{\ErrorTok}[1]{\textcolor[rgb]{0.64,0.00,0.00}{\textbf{{#1}}}}
\newcommand{\NormalTok}[1]{{#1}}
\usepackage{longtable,booktabs}
\usepackage{graphicx,grffile}
\makeatletter
\def\maxwidth{\ifdim\Gin@nat@width>\linewidth\linewidth\else\Gin@nat@width\fi}
\def\maxheight{\ifdim\Gin@nat@height>\textheight\textheight\else\Gin@nat@height\fi}
\makeatother
% Scale images if necessary, so that they will not overflow the page
% margins by default, and it is still possible to overwrite the defaults
% using explicit options in \includegraphics[width, height, ...]{}
\setkeys{Gin}{width=\maxwidth,height=\maxheight,keepaspectratio}
\setlength{\parindent}{0pt}
\setlength{\parskip}{6pt plus 2pt minus 1pt}
\setlength{\emergencystretch}{3em}  % prevent overfull lines
\providecommand{\tightlist}{%
  \setlength{\itemsep}{0pt}\setlength{\parskip}{0pt}}
\setcounter{secnumdepth}{5}

%%% Use protect on footnotes to avoid problems with footnotes in titles
\let\rmarkdownfootnote\footnote%
\def\footnote{\protect\rmarkdownfootnote}

%%% Change title format to be more compact
\usepackage{titling}

% Create subtitle command for use in maketitle
\newcommand{\subtitle}[1]{
  \posttitle{
    \begin{center}\large#1\end{center}
    }
}

\setlength{\droptitle}{-2em}
  \title{Policy Reverberation and Electricity Prices in the United States}
  \pretitle{\vspace{\droptitle}\centering\huge}
  \posttitle{\par}
  \author{Emma Heffernan \& Devvart Poddar}
  \preauthor{\centering\large\emph}
  \postauthor{\par}
  \predate{\centering\large\emph}
  \postdate{\par}
  \date{December 14, 2016}


% Redefines (sub)paragraphs to behave more like sections
\ifx\paragraph\undefined\else
\let\oldparagraph\paragraph
\renewcommand{\paragraph}[1]{\oldparagraph{#1}\mbox{}}
\fi
\ifx\subparagraph\undefined\else
\let\oldsubparagraph\subparagraph
\renewcommand{\subparagraph}[1]{\oldsubparagraph{#1}\mbox{}}
\fi

\hypersetup{colorlinks=true, citecolor=blue, linkcolor=black}
\usepackage{setspace}
\onehalfspacing
\usepackage{fancyhdr}
\pagestyle{fancy}
\fancyhead[OL,ER]{\large Policy Reverberation and Electricity Prices in the United States}
\fancyhead[EL,OR]{\thepage}
\fancyfoot[L]{\small Emma Heffernan \& Devvart Poddar}
\fancyfoot[R]{\small Hertie School of Governance}
\fancyfoot[C]{}

\begin{document}
\maketitle

{
\setcounter{tocdepth}{2}
\tableofcontents
}
\pagebreak

\section{Introduction}\label{introduction}

Rising electricity costs in developed nations has become the political
topic du jour across OECD jurisdictions. The cost of electricity
matters: the impact of rising electricity costs is felt most acutely by
low-income individuals, who see a greater proportion of their pay checks
directed towards their electricity bills. Given the economic
difficulties caused by rising electricity prices, this is an ongoing
topic of importance for academics and politicians alike.

Much of the blame for rising electricity prices has been levied against
government renewable energy policies
\citep[see][]{Ferguson2016, Lynch2016, Svaldi2016}. This perspective
claims that the requirement that electricity generation is made up of
more expensive renewable energy production will lead to higher costs.
Indeed, as the U.S. Energy Information Administrative (EIA) claims that
generation is responsible for 65\% of average electricity prices,
distribution accounts for 25\% of average electricity prices, and
transmission accounts for 9\% of electricity prices \citep{EIA2015}, a
rise in the cost of generating electricity may lead to increased prices.
Indeed, Hongbo Wang \citep{Wang2016} has found a relationship between
the first enactment of particular renewable energy policies and rising
energy costs.

The question that is left unanswered here is why there is a divergence
between states that have implemented renewable energy policies (knowing
that it may lead to increased electricity cost) and those that have not.
In part, this gap in the literature is because much research has been
limited to the impact of state level policies on electricity prices,
either through process-tracing extreme case studies
\citep[see][]{Netherlands2013} or by analyzing the relationship between
cost and policy at the state level \citep[see][]{Wang2016}. These
research efforts are invaluable; they provide an understanding of how
outliers have implemented renewable energy policies, and of general
trends assuming monolithic political contexts. However, this paper will
attempt to answer the question of what happens to renewable energy
policy, and subsequently electricity costs, when political environments
change. To address this question, this paper will conduct a textual
analysis of federal energy court rulings. Specifically, it will analyze
whether changes in the environmental language used in the rulings of the
federal energy courts is correlated with an increased electricity costs
at the state level. The United States is the case study in this analysis
because federal judicial rulings effects multiple different contexts.
This ``most different systems design'' enables us to analyze whether
changes in judicial opinion lead to changes in renewable energy policies
and in electricity prices.

\section{Literature Review: Two-Level Games, Renewable Energy Policies,
and Electricity
Prices}\label{literature-review-two-level-games-renewable-energy-policies-and-electricity-prices}

\subsection{Theoretical Framework: Two-Level
Games}\label{theoretical-framework-two-level-games}

The literature that concentrates solely on the link between state
renewable energy policies and cost ignores any impact that federal body
positions on renewable energy would have on state-level policy. This
assumption follows the realist trajectory. Traditional realist scholars
argue that there is no connection between judicial court rhetoric and
state-level action because the rhetoric is epiphenomenal
\citep{Mearsheimer1994} and governments enact renewable energy policies
when it suits their best interests \citep{Downs1996}. Thus, the realist
academic scholarship emphasizes that federal court-level rhetoric has no
concrete impact on the renewable policies of states. Further, a realist
would argue that if a state?s policies are in line with federal rulings,
it is primarily due to the self-interest of the state \citep{Waltz1979}.
As such, even in a situation where a state did adhere to the policies of
a treaty, the fact that a state did so is not because of any relevant
federal rhetoric.

On the other side of the ideological spectrum, constructivists argue
that states do try to adhere to ideological rhetoric. Constructivists
would argue that pressure from NGOs and politicians can alter the way
that states approach renewable policy \citep{Finnemore1996}. Finally, a
third perspective would argue that the reason why states implement
renewable energy policy is because of local pressure and local support
for specific policies. In this understanding, political institutions,
interest groups, and state actors determine whether a government
implements renewable energy policy \citep{Hafner2005}. Thus, whether
states actually comply depends on a more localized mobilization.

Clearly these three mainstream arguments point in different directions.
One maintains that federal court rhetoric has no effect on a states
renewable energy, while the other argues that the socialization that can
result from federal court decisions can result in a states changed
renewable energy policies. The third maintains that the federal court is
irrelevant, and that the renewable energy policies of states governments
result from local pressure. However, all three of these perspectives
have an incomplete focus. The way the three theories arrive at their
arguments is centred on a state-centric response to specific pressures.
Both constructivist and realist theories overlook the impact of domestic
civil society in either the deterioration or amelioration of a states
renewable energy policies \citep{Hafner2005}. The third perspective,
however, cannot explain why states implement renewable energy policies
in the absence of domestic mobilization. As such, all three theories
overlook a pertinent arena (whether state-centric or federal) that
impacts the implementation of renewable energy policies in specific
states.

Due to the limitations of these perspectives, this paper will follow a
theoretical framework that emphasizes the necessity of considering both
federal and state-centered influences on a states domestic policies.
This theoretical framework comes from Robert Putnam \citep{Putnam1988}s
concept of a two-level game. Putnam argues that politicians consider
both international and domestic dynamics in the formulation of
international agreements. Putnams focus intentionally deviates from
state-centric theories, which he argues are an uncertain foundation for
theorizing how domestic and international politics interact \citep[
p.433]{Putnam1988}. He further theorizes a concept of reverberation.
Putnam argues that \emph{international pressures can reverberate within
domestic politics} subsequently altering the position of domestic civic
society groups \citep[ p.454]{Putnam1988}. Building on the significance
of reverberation, Putnam argues that if the two spheres are linked
synergistically then they cannot be modeled separately \citep[
p.456]{Putnam1988}. Thus, the incompleteness of the aforementioned
theories is because they are attempting to model the influence of one
sphere in absence of a discussion of the impact of the other sphere. We
will use a modified version of this theory to model how the federal and
state spheres interact in the United States. Putnams theory illustrates
how the impact of federal court decisions can reverberate, and
potentially influence, the attitudes of the state legislature

\subsection{The Formulation of Energy Policy in the United
States}\label{the-formulation-of-energy-policy-in-the-united-states}

There are two court systems that ensure that electricity companies
comply with either the federal or government policies. The first court
system is the Federal Energy Regulatory Commission (FERC). FERC is an
independent regulatory agency within the United States Department of
Energy. Specifically, FERC regulates the transmission of electricity in
interstate commerce, and licenses non-federal hydropower projects. Their
jurisdiction also includes overseeing the transmission and wholesale
sales of electricity in interstate commerce, reviewing the siting
applications for electric transmission projects, reviewing mergers and
acquisitions by electricity companies, licenses and inspects private,
municipal, and state hydroelectric projects, and overseeing
environmental matters related to natural gas and hydroelectricity
projects \citep{FERC2016}. In contrast, the state-level adjudicative
bodies are responsible for the public utilities in the separate states.
State-level bodies are primarily concerned with regulating retail
electricity sales to customers, overseeing the construction of oil
pipelines, and regulation of activities of the municipal power systems
and most rural electric cooperatives \citep{FERC2016}.

These courts are guided by policies made at the state and federal level
regarding electricity generation and distribution. Energy policy in the
United States is formulated at both the federal and the state level.
Federal level policies tend to be more financially based, and rely
primarily on Tax Credits and Cost Recovery Systems. In contrast, state
level policies can take a wide variety of forms including
feed-in-tariffs, and renewable energy requirements \citep{FERC2016}.
Hence, it is at the state level that there is significant divergence in
policy as all states are subject to the overarching policies made at the
federal level (and, in tandem, all rulings that are made by FERC).

\subsubsection{Renewable Energy Policies, and Associated
Costs}\label{renewable-energy-policies-and-associated-costs}

Non-hydro renewable electricity production accounts for 2\% of total
electricity production in the United States; fossil fuels used for
electricity generation account for 40\% of all carbon dioxide emissions
resulting from human activity in the United States \citep{EIA2015}. The
overwhelming amount of carbon dioxide produced by electricity production
has prompted multiple states to encourage the increase of renewable
energy production feeding into the electricity grid through renewable
energy promotion policies \citep{Wang2016}.

Renewable energy promotion policies are separated in the literature into
two broad policies realms: market-pull policies and technological-push
policies. Market-pull policies aim at \emph{increasing renewable energy
use by creating demand for Renewable Energy Targets (RETs)}
\citep[p.~16]{EIA2015}. These policies include carbon taxation
strategies, technology and performance standards, and investment
promotion. Technology-push policies include public R\&D spending, tax
credits, and support for education and training. This paper will focus
specifically on market-pull policies that incentivize the generation of
renewable energy. There are two types of policies that incentivize the
use of renewable energy generation in the electricity grid. The first is
a market-based construction of generation incentives, and the second is
generation promotion policies. These policies are summarized in the
following table.

\begin{longtable}[c]{@{}lll@{}}
\toprule
Policy & Price.Driven & Quantity.Driven\tabularnewline
\midrule
\endhead
Generation Incentives & Feed-in tariffs & Tendering systems for long
term contracts Energy\tabularnewline
& Premium feed-in tariffs & Portfolio standards (quotas)\tabularnewline
Generation Promotion & Green Tariffs &\tabularnewline
\bottomrule
\end{longtable}

\subsubsection{Energy Portfolio standards
(quotas)}\label{energy-portfolio-standards-quotas}

In the United States, 29 states have implemented mandatory Renewable
Portfolio Standards (RPS), and a further 8 states have adopted
non-binding goals for renewable energy standards \citep{Wang2016}. These
states place an obligation on electricity supply companies to produce a
specific faction of their electricity from renewable sources. Though it
has been argued that RPS policies increase the amount of renewable
energy generation in a state \citep{Carley2009} critics argue that RPS
increases retail electricity prices because of the required extra
investment or costs involved with a switch to using renewable energy
\citep{Lesser2013}.

Previous studies of RPS policy effects have focused on evaluating
whether the implementation of policies leads to forecasted changes in
electricity costs either at the national level or the state level
\citep[see][]{PalmarBurtraw2005, Kydes2007, Wang2016}. Notably, Hongbo
Wang (2016) uses a differences-in-differences model to compare the
electricity prices of those states that have not implemented RPS to the
electricity prices of those states that have implemented RPS, and found
that implementation increases electricity prices when the RPS policy
first becomes binding. Her argument is persuasive, both in terms of the
evidence and methodology used. We want to expand on this argument by
asking why particular states have chosen to implement renewable energy
policies if they had the understanding that it may increase costs. We
will link the implementation of renewable energy policy to the
reverberation effects of particular voting patterns, and to the rhetoric
used by federal energy courts.

\subsubsection{Feed-in Tariffs, Tendering systems and Green
Tariffs}\label{feed-in-tariffs-tendering-systems-and-green-tariffs}

Despite the wide international use of FITS, the utilization of feed-in
tariffs have not been popular in the United States at the state level.
Currently, there are five states that have mandated FITs by law or
regulation, and a handful of states have voluntary FIT programs
\citep{EIA2013}. Given the lack of variation at the state level for this
policy in the United States, this policy will not be the main policy of
investigation.

In the United States, the use of tendering systems for long-term
contracts and green tariffs are not an independent policy measure in and
of themselves. Rather, companies can choose to engage in long-term
contracts or green tariffs in order to ensure that they meet the
requirements set out by other policies \citep{Hong2015}.

\section{Methodology}\label{methodology}

The most different systems design evaluates whether states that are
different in as many factors as possible have a similar dependent
variable of interest because of a change in the dependent variable. For
our analysis, the dependent variable is electricity pricing and the
implementation of RPS policies at the state level. We wonder if changes
in judicial rhetoric at the federal level impacts these two variables.

However, our states are different in many ways. Thus, we must control
for several variables to ensure that we are not overemphasizing the
impact of judicial rhetoric on electricity prices and RPS policy. For
example, variables such as political leanings, or whether the state
votes democrat or republican, may be correlated with different odds of
implementing RPS policies. Further, weather patterns, such as
particularly hot summers, may make electricity more expensive because of
increased demand. Finally, electricity market deregulation may impact
electricity prices regardless of whether the state has engaged in RPS
policies. These variables and their data sources are specified in the
following table:

\subsection{Variable Snapshot}\label{variable-snapshot}

\begin{longtable}[c]{@{}llllllll@{}}
\toprule
X & Prices & RPS & Deregulation & Freq..of.Order & Freq..of.Renewable &
Context..Renewable & Temperature\tabularnewline
\midrule
\endhead
Mean & 10.73 & 0.43 & 0.33 & 1.52 & 0.15 & 0.11 & 91.97\tabularnewline
Median & 10.16 & 0 & 0 & 1.41 & 0.08 & 0.08 & 89.5\tabularnewline
Min & 4.95 & 0 & 0 & 0.11 & 0.01 & 0 & 2\tabularnewline
Max & 23.08 & 1 & 1 & 2.51 & 0.42 & 0.31 & 185\tabularnewline
Unit & USD & - & - & \% point & \% point & \% point &\tabularnewline
Source & EIA & Wang (2016) & electricchoice.com & FERC (PDFs) & FERC
(PDFs) & FERC (PDFs) & NOAA\tabularnewline
\bottomrule
\end{longtable}

This paper relies on the assumption that reverberation of FERC decisions
results from three intertwined variables. The first is frequency. We
argue that the more frequently FERC uses particular terms, the more
likely the court decisions are to reverberate in the state context. This
is because the more a federal court makes statements about a topic, the
more likely this topic will arguably stay in the mind of a state
politician or state advocates. Repetition, here, is key for getting into
the long-term memory of particular individuals.

The second variable is that of renewable terminology. We argue that the
court can choose to use particular words in order to make a ruling. The
choice of some terms are more likely to encourage a positive association
with the use of renewable energy policies. Hence, the terms that will be
searched for as representative of renewable terminology are green,
renewable, hydro, sustainable, sustain, and environment. These terms
carry positive connotations with regards to renewable energy. The third
variable is that of action terminology. Here, words that indicate that
the court is ordering a particular corporation or person to take action
are correlated with situations where the individual does not have a
choice as to whether they will adhere to the ruling. The action terms
that will be searched for are order, revoke, and comply.

\subsection{Web-Scrapping}\label{web-scrapping}

Each of the relevant variables was obtained using web scrapping. The
variables were then matched based on year, month, and state in question.
For variables that did not change over the course of a year, or between
years, such as the proportion of the popular vote that went to each
party in the relevant elections, the same numbers were applied to all
observations until they changed. Hence, the votes for political parties
from 2000 were counted as the same for all months in 2000, 2001, 2002,
and 2003. Those from 2004 were counted for all months in 2004, 2005,
2006, and 2007, and so on.

\subsection{Textual Analysis}\label{textual-analysis}

The text obtained from the FERC online commissioners was analyzed using
a program called lemmatize. Lemmatize works through cleaning the text in
three distinct ways. 1. All punctuation and numbers are removed. 2. All
commonly used linking terms such as \emph{and }, \emph{or}, \emph{but}
are removed. 3. All terms are reduced to their root form. For example,
all terms such as compliance, comply, and complying would become to be
comply. This change enabled us to search only for the term comply
without being concerned that we were missing particular conjugations of
the word comply in our search.

\includegraphics{R_code_for_knitr_files/figure-latex/copy2-1.pdf}

These electricity prices can also be broken down by month, where it is
revealed that they tend to be lowest in the summer, and highest in the
summer.

\begin{verbatim}
## Warning: Removed 2 rows containing missing values (geom_pointrange).
\end{verbatim}

\includegraphics{R_code_for_knitr_files/figure-latex/copy3-1.pdf}

\subsection{Terminology Regression}\label{terminology-regression}

The terminology regression was conducted using a Fixed Effects
regression. The reason why the model relies on a fixed effects estimator
is because fixed effects controls for heterogeneity between cases over
time. We regressed the impact of terminology on prices in three ways.

First, we evaluated whether there was a correlation between the
frequency of order terminology and pricing. Second, we evaluated whether
there was a correlation between the frequency of renewable terminology
and pricing.

However, these analyses do not account for whether there is a connection
between the use of order terms and renewable terms. Specifically, the
word \emph{order} could be used 50 times in a document, but that does
not mean that it will impact the price of electricity unless it is
correlated with the use or production of renewable energy policy. Hence,
we created a third variable called context to measure the impact of the
two words occurring together on electricity prices. The context variable
counts the incidence of times that the order words and renewable words
are used together. So, for example, if the judge uses an order word and
uses a renewable word that within 30 words of the order word, this would
count as one instance for our third variable. The reason why we used 30
words as the window of distance between the terms is because the average
sentence length if 15-25 words, and often does not exceed 35 words.
Thus, if the order and renewable words are within 30 words of each
other, either they are in the same sentence, or they are in adjacent
sentences. This proximity increases the likelihood that they are used to
relay the same judicial idea.

\section{Results}\label{results}

\subsection{Terminology Regression}\label{terminology-regression-1}

The following chart presents the results of our analyses for the impact
of order, renewables and context terminology on electricity pricing. It
also includes the impact of RPS policies on electricity prices. It
should be noted that for each regression, we have controlled for the
other noted variables, as well as political affiliation, deregulation,
and weather.

The first regression, the bars on the left hand side, contain the
results for the base fixed effects regression. It should be noted that
there is a positive effect for all the variables. This implies that the
increased frequency (terminology) or enactment of RPS policies is
correlated with an increased price of electricity, when unobserved
heterogeneity is controlled for. These results, and in particular the
result that RPS policy is correlated with higher electricity prices, is
expected based on the theoretical framework and literature review.

However, as illustrated above, the prices of electricity tend to shift
with weather patterns that occur within a year. Hence, we include a
second regression that regresses based on month and year fixed effects.
Curiously, this regression has notable effects on the impacts of the
terminology and RPS variables. Specifically, both context and renewable
variables drop to a negative effect. The frequency of these variables is
now correlated with a decrease in the price of electricity. This means
that as the word renewable or the appearance of both order and renewable
words within 30 words of each other increases in frequency, the
correlated cost of electricity goes down. This result was not predicted
by our models, and is suggests that there must be some unspecified
bridging variables between the implementation of renewable energy
policies and increased electricity costs. These bridging variables could
include companies buying into long-term contracts that lock-them in to
more expensive renewable energy production. However, this is subject for
future research.

In contrast, order remains positively correlated with electricity prices
(though it is just above the 0.00 coefficient). RPS policies are also
correlated with increased electricity prices, but the impact is much
less than the effects that have been noted in past studies.

\begin{Shaded}
\begin{Highlighting}[]
\NormalTok{final <-}\StringTok{ }\KeywordTok{import}\NormalTok{(}\StringTok{"Output/processed data/mergeddata.csv"}\NormalTok{)}
\KeywordTok{ggplot}\NormalTok{(}\KeywordTok{subset}\NormalTok{(final, Judicial %in%}\StringTok{ }\KeywordTok{c}\NormalTok{(}\StringTok{"order.ave"}\NormalTok{, }\StringTok{"renewable.ave"}\NormalTok{, }\StringTok{"context_30"}\NormalTok{)),}
    \KeywordTok{aes}\NormalTok{(}\DataTypeTok{x =} \NormalTok{judicial, }\DataTypeTok{y =} \NormalTok{price,  }\DataTypeTok{colour =} \KeywordTok{interaction}\NormalTok{(Judicial, Prices))) +}\StringTok{ }\KeywordTok{facet_wrap}\NormalTok{( ~}\StringTok{ }\NormalTok{judicial) +}\StringTok{ }\KeywordTok{stat_summary}\NormalTok{(}\DataTypeTok{fun.y=}\StringTok{"mean"}\NormalTok{)}
    \KeywordTok{geom_point}\NormalTok{(}\DataTypeTok{alpha =} \FloatTok{0.3}\NormalTok{, }\DataTypeTok{position =} \StringTok{"jitter"}\NormalTok{) +}
\StringTok{    }\KeywordTok{geom_boxplot}\NormalTok{(}\DataTypeTok{alpha =} \DecValTok{0}\NormalTok{, }\DataTypeTok{colour =} \StringTok{"black"}\NormalTok{)+}
\KeywordTok{ggtitle} \NormalTok{(}\StringTok{"Judicial Rulings and Price"}\NormalTok{)}
\end{Highlighting}
\end{Shaded}

\subsection{Two Step}\label{two-step}

\section{Conclusion and Avenues for Further
Research}\label{conclusion-and-avenues-for-further-research}

The main finding of our analysis is that there is a pricing
contradiction when evaluating state policies and federal rulings. This
finding runs against the arguments in the mainstream literature that
suggest that RPS policies increase electricity prices because the
generation of electricity is more expensive if done through renewable
forms. However, if federal rulings that contain terminology around
ordering companies take specific action with regards to renewable energy
are correlated with reduced electricity prices, it may not all be about
the costs of electricity generation. Hence, further research would do
well to expand on this contradiction by utilizing process-tracing
analysis of the link between RPS and increased electricity costs in
select states. Specifically, what about RPS is driving prices upwards?
If it is not the cost of generating electricity, what is the economic
impact of implementing RPS?

Finally, our main judicial link here was federal court rulings. In
short, this variable was chosen in part to evaluate reverberation
effects, but also because of the relative ease of scrapping the data
from the FERC website (in comparison to from the websites of 50 state
judicial commissions). What remains to be seen, however, is whether
increased federal court focus on renewable energy is correlated with
increased state-level court focus on renewable energy. Further studies
would do well to evaluate the terminology used by courts at the
state-level, in order to analyze if there is a reverberation of federal
court rhetoric at the state level as well.

\pagebreak
\fancyhf{}

\renewcommand\refname{References}
\bibliography{Bibtex.bib}

\end{document}
